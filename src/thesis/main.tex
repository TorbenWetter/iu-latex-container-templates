%! Author = Torben Wetter

\documentclass[11pt]{article}

\usepackage[
  bibliographyfile={main.bib},
  graphicsdir={images/},
  logofile={logo.png},
  type={Bachelorarbeit},
  university={Internationale Hochschule Duales Studium},
  degree={Informatik B.\,Sc.},
  title={Beispiel-Thesis},
  subtitle={Eine LaTeX-Vorlage für Haus- und Abschlussarbeiten an der IU},
  author={Torben~Wetter},
  matriculationnumber={123456789},
  address={Musterstraße~7\\20355~Hamburg},
  supervisor={Prof.~Dr.~Musterfrau},
  submissiondate={31.03.2025}
]{iuthesis}

\begin{document}

\makecover

\maketoc

\section{Einleitung}
\label{sec:einleitung}

Hier steht die Einleitung.

Grundsätzlich werden alle gängigen \LaTeX-Befehle unterstützt.
Mathematische Formeln können im Fließtext eingebettet werden, \zB \(E = mc^2\).
Literatur sollte aus Citavi oder Zotero -- und dabei ist es wichtig, dass das BibTeX-Format verwendet wird -- exportiert und in der Datei \texttt{main.bib} gespeichert werden.
Anschließend sind Zitate möglich \autocite[S.\,2--5]{haider_realization_2021}.
Diese Quelle stammt zum Beispiel von den Autor:innen \citeauthor{haider_realization_2021}.
Ziemlich \enquote{easy peasy}, oder?

\begin{figure}[h]

  \centering

  \includegraphics[width=0.25\textwidth]{images/image.jpg}

  \caption[Ein lustiges Bild]{Ein lustiges Bild \autocite{illisioun_latex_2021}}

  \label{fig:ein-lustiges-bild}

\end{figure}

Das Einbinden von Bildern wie \autoref{fig:ein-lustiges-bild} ist ebenfalls möglich.
Auch auf Kapitel kann verwiesen werden, \zB \autoref{sec:theoretische-fundierung}.
100~\% der Studierenden, die diese Vorlage verwendet haben, haben ihre Thesis bestanden;
\dash Du wirst es auch schaffen!

\clearpage

\section{Theoretische Fundierung}
\label{sec:theoretische-fundierung}

Hier steht die theoretische Fundierung.

\subsection{Unterkapitel}
\label{subsec:unterkapitel}

Hier steht ein Unterkapitel.

\subsubsection{Unterunterkapitel}
\label{subsubsec:unterunterkapitel}

Hier steht ein Unterunterkapitel.

Tabellen sind gut geeignet, um Daten zu präsentieren.

\begin{tabularx}{\columnwidth}{p{2.75cm}lX}
  \textbf{Zeitraum}                     & \textbf{Umfang} & \textbf{Aufgabe}                                                                    \\ \toprule

  Vorbereitung: \newline bis 07.08.2023 & 10 Stunden      & Einlesen in Basisliteratur und Recherche zu den neuesten Trends im Bereich der LLMs \\ \midrule

  1. Woche:     \newline ab 07.08.2023  & 6 Stunden       & Aufsetzen des Softwareprojekts und erstes Prototyping, Überarbeitung der Einleitung \\ \midrule

  2. Woche:     \newline ab 14.08.2023  & 6 Stunden       & Literatur- und Marktrecherche, Analyse existierender Lösungen                       \\ \midrule

  3. Woche:     \newline ab 21.08.2023  & 10 Stunden      & Überarbeitung und Erweiterung der Theoretischen Fundierung                          \\ \midrule

  4. Woche:     \newline ab 28.08.2023  & 8 Stunden       & Verbesserung des Methodikkapitels, Vorbereitung der Entwicklung                     \\ \midrule

  5. Woche:     \newline ab 04.09.2023  & 16 Stunden      & Technische Umsetzung der Software, Präsentation der Forschungsergebnisse            \\ \midrule

  6. Woche:     \newline ab 11.09.2023  & 12 Stunden      & Bewertung der Software und Interpretation der Antworten des Chatbots                \\ \midrule

  7. Woche:     \newline ab 18.09.2023  & 6 Stunden       & Beantwortung der Leitfrage, Zusammenfassung der Ergebnisse im Fazit und Ausblick    \\ \midrule

  8. Woche:     \newline ab 25.09.2023  & 6 Stunden       & Plagiatsanalyse, Korrekturlesen und letzte Ausbesserungen
\end{tabularx}

\subsubsection{Weiteres Unterunterkapitel}
\label{subsubsec:weiteres-unterunterkapitel}

Hier steht ein weiteres Unterunterkapitel.

\subsection{Weiteres Unterkapitel}
\label{subsec:weiteres-unterkapitel}

Hier steht ein weiteres Unterkapitel.

\subsubsection{Noch ein Unterunterkapitel}
\label{subsubsec:noch-ein-unterunterkapitel}

Hier steht noch ein Unterunterkapitel.

\subsubsection{Und noch eins}
\label{subsubsec:und-noch-eins}

Und hier steht noch ein Unterunterkapitel.

\clearpage

\section{Methodik}
\label{sec:methodik}

Hier steht die Methodik.

\clearpage

\section{Präsentation und Analyse der Forschungsergebnisse}
\label{sec:praesentation-und-analyse-der-forschungsergebnisse}

Hier steht die Präsentation und Analyse der Forschungsergebnisse.

\clearpage

\section{Auswertung der Forschungsergebnisse}
\label{sec:auswertung-der-forschungsergebnisse}

Hier steht die Auswertung der Forschungsergebnisse.

\clearpage

\section{Fazit}
\label{sec:fazit}

Hier steht das Fazit.

\clearpage

\makebibliography

\appendix

\section{Interviewleitfaden}
\label{sec:interviewleitfaden}

Dieses Kapitel ist Teil des Anhangs.

\clearpage

\end{document}
